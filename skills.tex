% vim: set textwidth=80:
% vim: set formatoptions+=m:
\chapter{\TeX 使用技巧}
\echapter{Skills of \TeX}
\begin{Sketch}
    本章主要给出本模板中有用的一些技巧,能为编写论文提供极大的方便。
\end{Sketch}
\section{编译}
\esection{compile}
\label{sec:compile}
    \XeLaTeX 可以很好的支持中文,引入中文字体极为方便,是中文\TeX 的优秀引擎,
    本模板使用\XeLaTeX 。

推荐论文每章放置在一个独立的tex文件中,然后使用\verb|\include|命令导入,为了提高
编译速度,可以在导言区使用 
\begin{lstlisting}
\includeonly{<tex file1>,<tex file2>,...}
\end{lstlisting}
来分段编译。

\section{字体}
\esection{font}
\label{sec:font}
    字体文件在目录font中,如果有需要也可以在模板文件
    dutthesis.cls 中修改字体部分,选择需要的字体。
    \begin{lstlisting}[language=TeX]
	\setmainfont[
	Path = ./font/,
	BoldFont=timesbd.ttf,
	ItalicFont=timesi.ttf,
	BoldItalicFont=timesbi.ttf
	]{times.ttf}


	% 修改中文字体族,增加黑体
	\setCJKmainfont[
	Path = ./font/,
	BoldFont=simhei.ttf,
	ItalicFont=simkai.ttf,
	BoldItalicFont=simfang.ttf
	]{simsun.ttc}
	\setCJKfamilyfont{zhsong}[Path = ./font/]{simsun.ttc}
	\newcommand{\song}{\CJKfamily{zhsong}}
	\setCJKfamilyfont{zhhei}[Path = ./font/]{simhei.ttf}
	\newcommand{\hei}{\CJKfamily{zhhei}}
	\setCJKfamilyfont{FZXiHei}[Path = ./font/]{FZXH1K.TTF}
	\newcommand{\xhei}{\CJKfamily{FZXiHei}}
	\setCJKfamilyfont{zhkai}[Path = ./font/]{simkai.ttf}
	\newcommand{\kai}{\CJKfamily{zhkai}}
	\setCJKfamilyfont{zhfs}[Path = ./font/]{simfang.ttf}
	\newcommand{\fs}{\CJKfamily{zhfs}}
	\setCJKfamilyfont{FZXingKai}[Path = ./font/]{FZXKK.TTF}
	\newcommand{\xkai}{\CJKfamily{FZXingKai}}
    \end{lstlisting}
当然为了方便也可以使用本机字体,nofont分支不包含字体文件,下载很快,本机使用字体
如上所示。



\section{交叉引用}
\esection{corss reference}
\label{sec:cross-reference}
在论文中往往需要对章节、图表、公式进行引用,\LaTeX 编写论文基本依赖于hyperref包
实现引用。

本模板中推荐使用\verb|\autoref{<the label>}| 交叉引用,其中autoref的相关设置位于
dutthesis-utf8.cfg中
\begin{lstlisting}[language=TeX]
    \renewcommand{\figureautorefname}{图}
    \renewcommand{\tableautorefname}{表}
    \renewcommand{\figurename}{图}
    \def\equationautorefname#1#2\null{方程#1(#2\null)}
    \def\chapterautorefname#1#2\null{第#1#2 章}
    \def\subsectionautorefname#1#2\null{#1#2 小节}
    \def\sectionautorefname#1#2\null{#1#2 节}
\end{lstlisting}
基于这些设置,可以很好的实现交叉引用,而且不需要作者对被引用对象的识别,计算机将
分别对章节、图表、公式加以不同的表述,如:
\begin{lstlisting}[language=TeX]
    \autoref{cha:format},
    \autoref{sec:advance},
    \autoref{fig:cmm},
    \autoref{Eq:31},
\end{lstlisting}
\autoref{cha:format},
\autoref{sec:advance},
\autoref{fig:cmm},
\autoref{Eq:31},
当然\verb|\ref \eqref|等方式依旧可以使用,只是相对不如autoref方便。

hyperref包可以通过命令 \verb|\hypersetup{× = ×}| 设置引用连接的格式,使用本模
板时,可以通过 \verb|\documentclass[hidelinks]{dutthesis}| 选择无格式的交叉
引用。

\section{参考文献}
\esection{reference}
\label{sec:reference}
本模板使用bibtex引用参考文献,请将需要引用的文献放置于bib文件中
在需要引用的地方使用\verb|\cite{<keys>}|引用(上标),可以使用
\verb|\citet{<keys>}| 实现行内引用。

在正文中使用Reference环境输出参考文献
\begin{lstlisting}[language=TeX]
\begin{Reference}
\bibliography{<bib file>}
\end{Reference}
\end{lstlisting}
请记得添加Reference环境!

本参考文献样式使用\hyperlink{https://github.com/zepinglee/gbt7714-bibtex-style}
{gbt7714-bibtex-style},该样式可定制性强,arXiv支持做的还可以。我并对其中的gbt7714-numberical.bst做了一定修改,有需要可根据文档修改
\begin{lstlisting}[language=TeX]
FUNCTION {load.config}
{
  #2 'citation.et.al.min :=
  #1 'citation.et.al.use.first :=
  #4 'bibliography.et.al.min :=
  #3 'bibliography.et.al.use.first :=
  #1 'uppercase.name :=
  #0 'terms.in.macro :=
  #0 'year.after.author :=
  #1 'period.after.author :=
  #0 'italic.book.title :=
  #1 'sentence.case.title :=
  #1 'link.title :=
  #1 'title.in.journal :=
  #0 'show.patent.country :=
  #1 'show.mark :=
  #0 'space.before.mark :=
  #1 'show.medium.type :=
  "slash" 'component.part.label :=
  #0 'short.journal :=
  #1 'italic.journal :=
  #1 'bold.journal.volume :=
  #0 'show.missing.address.publisher :=
  #1 'space.before.pages :=
  #1 'only.start.page :=
  #0 'wave.dash.in.pages :=
  #1 'show.urldate :=
  #0 'show.url :=
  #0 'show.doi :=
  #1 'show.preprint :=
  #0 'show.note :=
  #0 'show.english.translation :=
  #1 'end.with.period :=
}
\end{lstlisting}
效果见\hyperref[cha:reference]{参考文献}
\section{章节、目录以及相关杂项}
\esection{chapter,section, contents and others}
\label{sec:Chapter-section-contents-others}
本节包含章节、目录以及其格式设定等问题。
\subsection{章节}
\esubsection{chapter and section}%
\label{sub:chapter-section}
为了实现双语目录,新章节都需要引入响应的英语章节。
\begin{table}[h]
    \bicaption[tab:chapter-sections]{中英文章节}{中英文章节命令对照表格,中文名
	在autoref中也有设定,见\autoref{sec:cross-reference}。
    }{Tab.}{The command of Chinese and english chapter and sections}
    \centering
    \begin{tabular*}{0.85\textwidth}{@{\extracolsep{\fill}}ccc}
	\hline
     中文名& 中文命令 & 英文命令 \\ \hline
     章 & \verb|\chapter[toc name]{chapter name}| & \verb|\echapter{toe name}| \\ \hline
     节 & \verb|\section| & \verb|\esection| \\ \hline
     小节 & \verb|\subsection| & \verb|\esubsection| \\ \hline
    \end{tabular*}
\end{table}
英文标题用于生成英文目录,硕士不需要英文目录,因此可以不要英文章节命令。

\subsection{目录}
\esubsection{contents}%
\label{sub:contents}
中文目录使用\verb|\tableofcontents| 生成,英文目录\verb|\tableofengcontents|,
图表目录\verb|\tableoffigurecontents,\tableoftablecontents|(这个需要存在英文目
录才行),或者也可以使用\verb|\listoffigures|, \verb|\listoftable|。
实际上只有博士需要英文,图表目录。

\subsection{杂项}
\esubsection{others}%
\label{sub:others}
摘要使用环境abstract, 英文摘要englishabstract,同时abstract环境附带了关键词选项
\begin{lstlisting}[language=TeX]
   \begin{abstract}{keywords,...}
       .....
   \end{abstract} 
   \begin{englishabstract}{english keywords, ....}
       .....
   \end{englishabstract}
\end{lstlisting}

尽管学校的模板中没有对每章正文之前加入摘要,或者简述,遵照过往的经验,依旧定义了
一个Sketch环境来引入章节简述。

发表论文情况使用环境Publics。

致谢定义了环境 Acknowledgement。

博士论文需要有个人简历,使用环境Resume。
\begin{lstlisting}[language=TeX]
    \begin{Resume}
	\begin{minipage}{0.8\textwidth}
		\linespacing{1.6}
	    \begin{tabular}{lp{2\baselineskip}}
		姓名:  三分先生\\
		性别: 男\\
		出生年月:  1991 年 7 月  \\
		民族:  汉 \\
		籍贯:  灵台方寸山,斜月三星洞\\
		研究方向: \LaTeX  以及 \XeLaTeX\\
		简历:\\
	    \end{tabular}
	\end{minipage}
	\begin{minipage}{0.2\textwidth}
	    \flushright
	    \vspace*{-80pt}
		\fboxrule=1.2pt \fboxsep=0pt
		\fcolorbox{gray}{white}{
		     \includegraphics[width=3.5cm,height=5cm]{figures/doctor-hwzs.pdf}
	    }
    \end{minipage}

    三分先生不知何许人也,其人额大颡突,发线高耸,额雄踞头部三分,乃以三分为号。
    人或曰,昔者杜甫诗云,``功盖三分国,名称八卦图",
    如此先生岂非有孔明之大志,怀管乐之奇才乎?
    非也,乱世雄图,与治世而何?故无其才,安能谋事。
    天下当混为一,安可分!三分者,徒先生之怪貌者也。
    $\ldots \cdots$
    \end{Resume}
\end{lstlisting}

