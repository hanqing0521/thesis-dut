\chapter{正文格式说明}
\echapter{format}%
\label{cha:format}
“正文”不可省略。

正文是博士学位论文的主体,要着重反映研究生自己的工作,要突出新的见解,例如新思想、新观点、新规律、新研究方法、新结果等。正文一般可包括:理论分析;试验装置和测试方法;对试验结果的分析讨论及理论计算结果的比较等。

正文要求论点正确,推理严谨,数据可靠,文字精练,条理分明,文字图表清晰整齐,计算单位采用国务院颁布的《统一公制计量单位中文名称方案》中规定和名称。各类单位、符号必须在论文中统一使用,外文字母必须注意大小写,正斜体。简化字采用正式公布过的,不能自造和误写。利用别人研究成果必须附加说明。引用前人材料必须引证原著文字。在论文的行文上,要注意语句通顺,达到科技论文所必须具备的“正确、准确、明确”的要求。

\section{论文格式基本要求}
\esection{The format requirement of thesis}
论文格式基本要求:
\begin{asparaenum}[(1)]
    \item  纸  型:A4纸,双面打印。

    \item 页边距:上3.5cm,下2.5cm,左2.5cm、右2.5cm。

    \item 页  眉:2.5cm,页脚:2cm,左侧装订。

    \item 字  体:正文全部宋体、小四。

    \item 行  距:多倍行距:1.25,段前、段后均为0行,取消网格对齐选项。

    \item 对  齐:采用两边对齐。

    \item 软件要求:论文的撰写可以采用Microsoft word (2003以上版本)等主流文字编辑软件并便于生成PDF文档。
\end{asparaenum}
\section{论文页眉页脚的编排}
\esection{Header and Footer}
一律用阿拉伯数字连续编页码。页码应由引言首页开始,作为第1页。封一、封二和封底不编入页码。将摘要、Abstract、目录等前置部分单独编排页码。页码必须标注在每页页脚底部居中位置,宋体,小五。

奇数页页眉,宋体,五号,居中。填写内容为“大连理工大学博士学位论文”。

偶数页页眉,宋体,五号,居中。填写内容是论文的中文题目。                                                                                         

模板中已经将字体和字号要求自动设置为缺省值,只需双击页面中页眉位置,按要求将填写内容替换即可。
\section{论文正文格式}
\esection{Format of main body}
正文选用模板中的样式所定义的“正文”,每段落首行缩进2字;或者手动设置成每段落首行缩进2字,字体:宋体,字号:小四,行距:多倍行距 1.25,间距:前段、后段均为0行,取消网格对齐选项。

模板中已经自动设置为缺省值。

模板中的正文内容不具备自动调整格式的能力,如果要粘贴,请先粘贴在记事本编辑器中,再从记事本中拷贝,然后粘贴到正文中即可。或者使用手动设置,将粘贴内容的格式设置成要求的格式。

\section{章节标题格式}
\begin{asparaenum}[(1)]
    \item 
    每章的章标题选用模板中的样式所定义的“标题1”,居左;或者手动设置成字体:黑体,居左,字号:小三,1.5倍行距,段后1行,段前为0行。每章另起一页。章序号为阿拉伯数字。在输入章标题之后,按回车键,即可直接输入每章正文。

\item
 每节的节标题选用模板中的样式所定义的“标题2”,居左;或者手动设置成字体:黑体,居左,字号:四号,1.5倍行距,段后为0行,段前0.5行。

\item 
    每节的节标题选用模板中的样式所定义的“标题2”,居左;或者手动设置成字体:黑体,居左,字号:四号,1.5倍行距,段后为0行,段前0.5行。

\item
    节中的一级标题选用模板中的样式所定义的“标题3”,居左;或者手动设置成字体:黑体,居左,字号:小四,1.5倍行距,段后为0行,段前0.5行。 正文各级标题编号的示例如图2-1所示。
\end{asparaenum}

\section{各章之间的分隔符设置}
\esection{the seperator betwen chapters}
\LaTeX 的章节设置不需要过多的担心。 在这里我可以指出,为了实现某些页面只出现在奇数页,可以使用\textbackslash stcleardp命令。

\section{正文中的编号}
\esection{the number of main body}%
\label{sec:the_number_of_main_body}
正文中的图、表、附注、公式一律采用阿拉伯数字分章编号。

如图2.1,表3.3,附注4.5,式6.7等。如“图2.1”就是指本论文第2章的第1个图。文中参考文献采用阿拉伯数字根据全文统一编号,如文献[3],文献[3,4],文献[6-10]等,在正文中引用时用右上角标标出。附录中的图、表、附注、参考文献、公式另行编号,如图A1,表B2,附注B3,或文献[A3]。

\section{正文中内容要求}
\esection{Requirement of main body}

正文中的每章都要加入“引言”  这部分内容主要用来对于该章节的主要内容进行简述。

如在论文中要加入“定理与证明”部分,在该部分中要把论文中采用的定理,以及在论文中出现的证明过程写出来。

论文正文一般应在4~10万字。


\section{本章小结}
\esection{conclusion of this chapter}
\Large 样式:
\vspace{10pt}

