% vim: set textwidth=80:
% vim: set formatoptions+=m:
\chapter{不同学位情况说明}
\echapter{details for different degrees}
本模板优先支持大连理工大学博士学位论文。
当选择了不同学位,\verb|\@degree| 将会设定成学位类型,硕士、博士等等。
学校名在dutthesis.cfg中定义,\verb|\universityname|英文名也在其中。
(\universityname\makeatletter\@degree\makeatother)
\section{博士学位论文}
\esection{doctor thesis}
\begin{lstlisting}[language=TeX]
    \documentclass[doctor]{dutthesis} %其他选项也可添加
\end{lstlisting}

大连理工大学论文模板结构比较混乱,比如标题的格式,加粗不加粗都存在一定争议,本论文中居中标题没有加粗,如果有需要可以自己添加(没必要)。
此外应该注意博士论文中,独创性声明与版权声明在同一页,模板已经处理好。
其中独创性声明命令为\verb|\declare|,
对于博士论文来说,\verb|\declare|包含了版权声明\verb|\dutauthorization|命令(见
dutthesis-utf8.cfg)。

博士论文中英文摘要不进入目录,目录第一条为绪论,且需要加入英文目录,图、表目录。
本模板中博士论文可以依次使用命令
\begin{lstlisting}[language=TeX]
\tableofcontents
\tableofengcontents
\cleardoublepage
\tableoffigurecontents
\tableoftablecontents
\end{lstlisting}
其中 \lstinline|\cleardoublepage| 用于清空页面到奇数页,而
\lstinline|\tableofengcontents, \tableoffigurecontents, \tableoftablecontents|为
(重)新定义的各类目录,这里如果不使用英文目录,则图表目录也不能用上述命令,
但是 \lstinline|\listoffigures,\listoftables|依旧可以使用,会导致图表目录都另启
一页。

博士论文需要填写主要符号表这一页,范例如下:
\begin{lstlisting}[language=TeX]
\subcapter{主要符号}

\begin{table}[H]
%\bicaption[table]{主要符号表}{}
%{Tab.}{}
%根据学校给出的模板,主要符号表为单独一章,表头无序号,不计入“表目标”中  李2018.3
\begin{center}
\begin{tabular}{ccc}
  \hline
  % after \\: \hline or \cline{col1-col2}  ...
  符号          & 代表意义     & 单位  \\
  $T$   & 温度     & K \\
  $m$   & 质量     & kg \\
  $L$   & 长度     & m \\
  $\kappa$   & 光场耗散     & Hz \\
  $\omega$   & 频率     & Hz \\
  $\omega_{m}$   & 机械振子频率     & Hz \\
  $\gamma_{m}$   & 机械振子耗散率     & Hz \\
  $g$   & 光力耦合强度     & Hz \\
  $\hbar$   & 约化普朗克常数(如未特别强调,取$\hbar=1$)     & J$\cdot$ s  \\
  $k_B$  & 玻尔兹曼常数     & J/K \\
  $\langle\dots\rangle$   & 取量子态下期望值     & \\
  $\delta(\dots)$   & 狄拉克函数     & \\ 
  $\delta_{ij}$   & 克罗内克符号     & \\ 
  $H_{\textrm{eff}}$   & 有效哈密顿量     & \\ 
  $H_{\textrm{int}}$   & 相互作用哈密顿量     & \\ 
  $H_{\textrm{tot}}$   & 总哈密顿量     & \\ 
\end{tabular}
\end{center}
\end{table}
\end{lstlisting}

成果页面使用定制的Publics环境,定义代码如下:
\begin{lstlisting}[language=TeX]
    \newenvironment{Publics}[1][\@publicstitle]
    {%
      \stcleardp
      \phantomsection
      \addcontentsline{toc}{chapter}{\@publicstitletoc}
      \subchapter*{#1}
      \addcontentsline{toe}{chapter}{Achievements}
      \song\xiaosihao
    }{\par}
\end{lstlisting}
博士的科研成果页面将根据学位而自动修改。
主要配置为:
\begin{lstlisting}[language=TeX]
\newcommand{\@publicstitle}{攻读\@degree 学位期间科研项目及科研成果}
\newcommand{\@publicstitletoc}{攻读\@degree 学位期间科研项目及科研成果}
\end{lstlisting}

博士论文需要加入个人简介页面,模板重新定义了Resume环境。
范例如下:
\begin{lstlisting}[language=TeX]
 \begin{Resume}
    \begin{minipage}{0.8\textwidth}
	    \linespacing{1.6}
	\begin{tabular}{lp{2\baselineskip}}
	    姓名:  三分先生\\
	    性别: 男\\
	    出生年月:  1991 年 7 月  \\
	    民族:  汉 \\
	    籍贯:  灵台方寸山,斜月三星洞\\
	    研究方向: \LaTeX  以及 \XeLaTeX\\
	    简历:\\
	\end{tabular}
    \end{minipage}
    \begin{minipage}{0.2\textwidth}
	\flushright
	\vspace*{-80pt}
	    \fboxrule=1.2pt \fboxsep=0pt
	    \fcolorbox{gray}{white}{
		 \includegraphics[width=3.5cm,height=5cm]{figures/doctor-hwzs.pdf}
		 %插入照片
	}
\end{minipage}

三分先生不知何许人也,其人额大颡突,发线高耸,额雄踞头部三分,乃以三分为号。
人或曰,昔者杜甫诗云,``功盖三分国,名称八卦图",
如此先生岂非有孔明之大志,怀管乐之奇才乎?
非也,乱世雄图,与治世而何?故无其才,安能谋事。
天下当混为一,安可分!三分者,徒先生之怪貌者也。
$\ldots \cdots$
\end{Resume}   
\end{lstlisting}

\section{硕士学位论文}
\esection{master thesis}
\begin{lstlisting}[language=TeX]
    \documentclass[master]{dutthesis} %其他选项也可添加
\end{lstlisting}
硕士论文与博士稍有不同。

独创性声明页面不包括版权声明,版权声明在最后一页。

硕士论文只有中文目录,且中英文摘要都要进入目录。

硕士论文的Introduction定义了新环境,因为格式特殊,所以要在Introduction环境中写,
坑爹啊!!!

硕士论文有一章为结论,定义了Conclusion环境,博士模板无此环境!!!
\begin{lstlisting}[language=TeX]
%%%%%%%%%%%%%%%
%  结论环境,硕士需要  %
%%%%%%%%%%%%%%%
\ifmasterdegree
\newenvironment{Conclusion}
{%
  \stcleardp
  \phantomsection
  \addcontentsline{toc}{chapter}{结论}
  \subchapter*{结论}
  \addcontentsline{toe}{chapter}{conclusion}
  \song \xiaosihao
}{\par}
\fi
\end{lstlisting}

硕士论文发表情况页面自动根据 \lstinline|\@degree| 修改,选择硕士论文选项就好。

硕士论文的版权页面将自动生成,为最后一页。
\section{学士学位}
\esection{bachelor}
Waiting
